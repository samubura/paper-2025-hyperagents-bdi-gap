%% The first command in your LaTeX source must be the \documentclass command.
%%
%% Options:
%% twocolumn : Two column layout.
%% hf: enable header and footer.
\documentclass[
% twocolumn,
% hf,
]{ceurart}

%%
%% One can fix some overfulls
\sloppy

\usepackage{my-main-style}

%%
%% Minted listings support 
%% Need pygment <http://pygments.org/> <http://pypi.python.org/pypi/Pygments>
\usepackage{listings}
%% auto break lines
\lstset{breaklines=true}

%%
%% end of the preamble, start of the body of the document source.
\begin{document}

%%
%% Rights management information.
%% CC-BY is default license.
\copyrightyear{2025}
\copyrightclause{Copyright for this paper by its authors.
  Use permitted under Creative Commons License Attribution 4.0
  International (CC BY 4.0).}

%%
%% This command is for the conference information
\conference{HyperAgents'25: 2nd Workshop on Hypermedia Multi-Agent Systems,
  October 25-30, 2025, Bologna, Italy}

%%
%% The "title" command
\title{The Abstraction Gap Between BDI Agents and Hypermedia and What We Can Do About It}

% \tnotemark[1]
% \tnotetext[1]{You can use this document as the template for preparing your
%   publication. We recommend using the latest version of the ceurart style.}

%%
%% The "author" command and its associated commands are used to define
%% the authors and their affiliations.
\author[1]{Samuele Burattini}[%
orcid=11111-2222-3333-4444,
email=samuele.burattini@unibo.it,
]
\cormark[1]
\fnmark[1]

\author[1]{Martina Baiardi}[%
orcid=11111-2222-3333-4444,
email=m.baiardi@unibo.it,
]
\fnmark[1]

\author[1]{Giovanni Ciatto}[%
orcid=11111-2222-3333-4444,
email=giovanni.ciatto@unibo.it,
]


\author[1]{Danilo Pianini}[%
orcid=11111-2222-3333-4444,
email=danilo.pianini@unibo.it,
]


% \author[1]{Alessandro Ricci}[%
% orcid=11111-2222-3333-4444,
% email=a.ricci@unibo.it,
% ]

\address[1]{\disi, \unibo}


%% Footnotes
\cortext[1]{Corresponding author.}
\fntext[1]{These authors contributed equally.}

%%
%% The abstract is a short summary of the work to be presented in the
%% article.
\begin{abstract}
  Traditional BDI agents, 
  rooted in logic programming, 
  remain poorly integrated with the hypermedia nature of open Web environments
  which typically rely on Semantic Web technologies such as RDF and OWL.
  This paper examines the abstraction gap between these paradigms and surveys existing integration efforts on a conceptual and technical level.
  Our proposal for a deeper integration relies on a generalized BDI engine to enable the development of BDI agents that can directly reason and operate on hypermedia resources.
  We reflect on the potential benefits and challenges of this approach and show preliminary results of a proof-of-concept implementation.
\end{abstract}

%%
%% Keywords. The author(s) should pick words that accurately describe
%% the work being presented. Separate the keywords with commas.
\begin{keywords}
  BDI  \sep 
  Hypermedia \sep 
  ...
\end{keywords}

%%
%% This command processes the author and affiliation and title
%% information and builds the first part of the formatted document.
\maketitle

\section{Introduction}

\nocite{*} %REMOVE THIS

\section{Background}

\subsection{BDI Agents}

\sbnote{Add BDI agents background, e.g. BDI architecture, BDI logic, etc.}

\subsection{The Web and Hypermedia}

\sbnote{Add Web and Hypermedia background, e.g. Web architecture, Hypermedia as the engine of application state (HATEOAS), Semantic Web and Ontological Reasoning etc.}

\subsection{Hypermedia Multi-Agent Systems}

\sbnote{Very briefly introduce the research context of hMAS}

\section{Integrating \acs{BDI} Agents and Hypermedia}

\subsection{Gap Analysis}
\sbnote{What is needed and why there is a gap}

\subsection{Integration Requirements}
\sbnote{Ideally, a general list of requirements which guide the analysis of existing approaches and the design of the proposed approach.}
\sbnote{Since this is a short, we can also keep it short and consider a more in-depth analysis for the future...}

\subsection{Existing Approaches}
\sbnote{Related works, JASDL (Jason+Ontological reasoning), Yggdrasil framework, Hypermedea (Saint-Etienne Cartago Artifacts for Hypermedia), others.... }

\section{Levelling the Abstraction Gap with a Generalized BDI Engine}
\sbnote{Preliminary description of the approach + possible showcase of WIP prootype?}

\section{Discussion}

\sbnote{Challenges and opportunities of the approach, e.g. ontological reasoning and inference, Belief consistency, "goal" definition and management, }

\section{Conclusion}


%%
%% The acknowledgments section is defined using the "acknowledgments" environment
%% (and NOT an unnumbered section). This ensures the proper
%% identification of the section in the article metadata, and the
%% consistent spelling of the heading.
\begin{acknowledgments}
  \sbnote{Add Acks} 
\end{acknowledgments}

%% The declaration on generative AI comes in effect
%% in Janary 2025. See also
%% https://ceur-ws.org/GenAI/Policy.html
\section*{Declaration on Generative AI}

%   {\em Either:}\newline
%   The author(s) have not employed any Generative AI tools.
%   \newline
  
%  \noindent{\em Or (by using the activity taxonomy in ceur-ws.org/genai-tax.html):\newline}
%  During the preparation of this work, the author(s) used X-GPT-4 and Gramby in order to: Grammar and spelling check. Further, the author(s) used X-AI-IMG for figures 3 and 4 in order to: Generate images. After using these tool(s)/service(s), the author(s) reviewed and edited the content as needed and take(s) full responsibility for the publication’s content. 
\sbnote{TODO}

%%
%% Define the bibliography file to be used
\bibliography{bibliography}

%%
%% If your work has an appendix, this is the place to put it.
% \appendix

% \section{Online Resources}
% The sources for the ceur-art style are available via
% \begin{itemize}
% \item \href{https://github.com/yamadharma/ceurart}{GitHub},
% % \item \href{https://www.overleaf.com/project/5e76702c4acae70001d3bc87}{Overleaf},
% \item
%   \href{https://www.overleaf.com/latex/templates/template-for-submissions-to-ceur-workshop-proceedings-ceur-ws-dot-org/pkfscdkgkhcq}{Overleaf
%     template}.
% \end{itemize}

\end{document}

%%
%% End of file
